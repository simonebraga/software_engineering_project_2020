\documentclass[a4paper]{article}

\usepackage[utf8]{inputenc}

\usepackage{graphicx,float}
\graphicspath{ {./resources/} }

\usepackage{tabularx,enumitem}
\usepackage{booktabs}
\newcommand{\tabitem}{~~\llap{\textbullet}~~}

\setlength{\parindent}{0pt}

\usepackage[hidelinks]{hyperref}

\begin{document}

%----------------------------------------------------------------------------------------
%	TITLE PAGE
%----------------------------------------------------------------------------------------

\begin{titlepage}
	\newcommand{\HRule}{\rule{\linewidth}{0.5mm}}
	
	\center
	
	%------------------------------------------------
	%	Headings
	%------------------------------------------------
	
	\textsc{\LARGE Politecnico di Milano}\\[1.5cm]
	
	\textsc{\Large Computer Science and Engineering}\\[0.5cm]
	
	\textsc{\large Software Engineering 2}\\[0.5cm]
	
	%------------------------------------------------
	%	Title
	%------------------------------------------------
	
	\HRule\\[0.4cm]
	
	{\huge\bfseries SafeStreets\medskip\\
	\normalsize Design Document}\\[0.4cm]
	
	\HRule\\[1.5cm]
	
	%------------------------------------------------
	%	Author(s)
	%------------------------------------------------
	
	\begin{minipage}{0.4\textwidth}
		\begin{flushleft}
			\large
			\textit{Authors}\\
			\textsc{Simone Braga}\\
			\textsc{Juan Calderon}\\
			\textsc{Marzia Favaro}
		\end{flushleft}
	\end{minipage}
	~
	\begin{minipage}{0.4\textwidth}
		\begin{flushright}
			\large
			\textit{Professor}\\
			\textsc{Elisabetta Di Nitto}\linebreak\linebreak
		\end{flushright}
	\end{minipage}
	
	%------------------------------------------------
	%	Logo
	%------------------------------------------------
	
	\vfill\vfill\vfill
	
	\begin{figure}[H]
	\centering
	\includegraphics[width=0.5\textwidth]{polimi_logo}
	\end{figure}
	
	%------------------------------------------------
	%	Date
	%------------------------------------------------
	
	\vfill\vfill\vfill
	
	{\large December 9, 2019\\ Version 1.0} %TODO Check date and version for every new release
		%----------------------------------------------------------------------------------------
	
	\vfill
	
\end{titlepage}

%----------------------------------------------------------------------------------------

\newpage
\pagenumbering{roman}

\tableofcontents

\newpage
\pagenumbering{arabic}

\section{Introduction}

\subsection{Purpose}
\subsection{Scope}
\subsection{Definitions, Acronyms, Abbreviations}
\subsection{Revision history}
\subsection{Reference Documents}
\subsection{Document Structure}

\newpage
\section{Architectural design}

\subsection{Overview}
\subsection{Component view}
\subsection{Deployment view}
\subsection{Runtime view}
\subsection{Component interfaces}
\subsection{Selected architectural styles and patterns}
\subsection{Other design decisions}

\newpage
\section{User interface design}

\newpage
\section{Requirements traceability}
The design is thought to satisfy the requirements shown in the RASD. To clarify the motivations that led to the definitions of the components here are reported the links between requirements and components.

%G1

\begin{itemize}
\item
  \textbf{R1} When a picture is taken using SafeReports, a new violation
  record is generated.
    \subitem 
    %TODO need to specify also user, app, router...?
    \textbf{[REPORT MANAGER]} and \textbf{[PICTURE MANAGER]} are interested in the process. In particular [PICTURE MANAGER] sends the picture to the [REPORT MANAGER], which embeds the image with the other violation data.
  	
\item
  \textbf{R2} When a new violation record is generated, the current
  position of the user is added to the report.\\
  \textbf{R3} When a new violation record is generated, the timestamp is
  added to the report.
  \subitem
    \textbf{[REPORT MANAGER]} ensures to get the position and the timestamp from the picture data, which is sent to this component together with the image from \textbf{[PICTURE MANAGER]}.
  
\item
  \textbf{R4} When a new violation record is generated, the photo is
  scanned by an OCR software to automatically detect the plate.
    
   \subitem
      \textbf{[PICTURE MANAGER]} interacts with the OCR software to get the reading of the license plate and sends the result to \textbf{[REPORT MANAGER]}.
  
\item
  \textbf{R5} If the OCR software fails in detecting the plate, the user
  is notified and asked to repeat the procedure.\\
  \textbf{R6} If the OCR software detects the plate, the user is asked
  to confirm the violation report.
  \subitem
    In case of failure in the interpretation of the picture, \textbf{[REPORT MANAGER]} deals with the exception. The messages between this component and the user flow through the \textbf{[ROUTER]} and of course occur inside the \textbf{[APP]} (the only means of communication between users and the system).
  
\item
  \textbf{R7} If the user confirms the violation report, it is sent to
  SafeStreets.\\
  \textbf{R8} SafeStreets stores the information about the violation
  only if there aren't equivalent events already stored.
   \subitem
   \textbf{[REPORT MANAGER]} deals with the duplicated violations and eventually discards them. If it considers the violation valid and receives the approbation of the user, it sends the data to the \textbf{[DBMS]}, where it is stored.
  
\end{itemize}

%G2

\begin{itemize}
\item
  \textbf{R9} SafeAnalytics allows common users to get data about
  violations selecting zone, time and type of violation.
  \subitem	
	\textbf{[APP]} is aware of the type of user who is interacting with the system (thanks to \textbf{[USER MANAGER]}) and allows it to send (through the [ROUTER]) just the allowed filters to the \textbf{[QUERY HANDLER]}.
  
\item
  \textbf{R10} SafeAnalytics anonymizes information before sending it to
  common users.\\
%G3
  \textbf{R11} SafeAnalytics allows authorities to get all the information
  stored by SafeStreets.
  \subitem
    \textbf{[QUERY HANDLER]} deals with the data received and filters it for the common user, while letting all the informations available for the authorities.

\end{itemize}

%G4
\begin{itemize}
\item
  \textbf{R12} SafeStreets must store data about accidents provided by
  the municipality when available.
  \subitem
    \textbf{[INFO UPDATER]} manages the periodic update of the database adding the data acquired from the [MUNICIPALITY].
    
\item
  \textbf{R13} SafeStreets must analyze collected data crossed with data
  from the municipality to identify possible interventions.
  \subitem
    \textbf{[QUERY MANAGER]} gets the suggestions related to the user's query from \textbf{[DBMS]} %TODO data warehouse...
    
\item
  \textbf{R14} SafeSuggestions allows municipality users to get
  suggestions provided by SafeStreets.
  \subitem
  	\textbf{[APP]} is aware of the type of user and  allows municipality users to access SafeSuggestions' functionalities through \textbf{[ROUTER]}. 
\end{itemize}

\textbf{G5) SafeStreets must generate traffic tickets forwarding
reliable data to MTS.}

\begin{itemize}
\item
  \textbf{R15} When the users send a violation report, its integrity is
  checked.
\item
  \textbf{R16} If the integrity check is not successful, the violation
  report is discarded.
\item
  \textbf{R17} SafeStreets must forward every new stored violation
  report to MTS to generate traffic tickets.
\end{itemize}

\textbf{G6) SafeTickets must allow authorities to get statistics on
issued tickets.}

\begin{itemize}
\item
  \textbf{R18} When a new ticket is generated using MTS, ticket-related
  data are stored by SafeStreets.
\item
  \textbf{R19} SafeStreets must build statistics from stored data about
  issued tickets.
\item
  \textbf{R20} SafeTickets allows authorities to get information and
  statistics on issued tickets.
\end{itemize}


\newpage
\section{Implementation, integration and test plan}
\newpage
\section{Effort spent}
\section{References}

\end{document}